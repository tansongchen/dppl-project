\documentclass{beamer}
\title{Abstracting Distributed Computing with System $\mathsf{F_{<:}}$}
\subtitle{Project Final Report in \\ Course \emph{Design Principles of Programming Languages}}
% \titlegraphic{\includegraphics[width=.5\textwidth]{pyscf-logo.png}}
\author[Tan, Wang and Zhu]{Songchen Tan\thanks{tansongchen@pku.edu.cn}\quad Yue Wang\thanks{wangyue0502@pku.edu.cn}\quad Ruidong Zhu\thanks{zhurd@pku.edu.cn}}
\institute[PKU]{\normalsize Peking University}
\date{\today}
\useinnertheme{rectangles}
\useoutertheme{infolines}
\usecolortheme[rgb={0.9,0.4,0.2}]{structure}
\usecolortheme{rose}
\usecolortheme{whale}
\usefonttheme{professionalfonts}
\beamertemplatenavigationsymbolsempty
\newif\ifnottitlepage
\nottitlepagefalse
\setbeamercolor*{palette secondary}{use=structure,fg=white,bg=structure.fg!90!black}
\setbeamercolor*{palette tertiary}{use=structure,fg=white,bg=structure.fg!80!black}
\defbeamertemplate*{title page}{mytitlepage}[1][]{
  \vbox{}
  \vfill
  \begingroup
    \centering
    {\usebeamercolor[fg]{titlegraphic}\inserttitlegraphic\par}
    \vskip1em
    \begin{beamercolorbox}[sep=8pt,center,#1]{title}
      \usebeamerfont{title}\inserttitle\par%
      \ifx\insertsubtitle\@empty%
      \else%
        \vskip0.25em%
        {\usebeamerfont{subtitle}\usebeamercolor[fg]{subtitle}\insertsubtitle\par}%
      \fi%     
    \end{beamercolorbox}%
    \vskip1em\par
    \begin{beamercolorbox}[sep=8pt,center,#1]{author}
      \usebeamerfont{author}\insertauthor
    \end{beamercolorbox}
    \begin{beamercolorbox}[sep=8pt,center,#1]{institute}
      \usebeamerfont{institute}\insertinstitute
    \end{beamercolorbox}
    \begin{beamercolorbox}[sep=8pt,center,#1]{date}
      \usebeamerfont{date}\insertdate
    \end{beamercolorbox}
  \endgroup
  \vfill
}
\setbeamertemplate{title page}[mytitlepage]

\defbeamertemplate*{headline}{myheadline}
{
  \begin{beamercolorbox}[colsep=1.5pt]{upper separation line head}
  \end{beamercolorbox}
  \begin{beamercolorbox}{section in head/foot}
    \vskip2pt\insertnavigation{\paperwidth}\vskip2pt
  \end{beamercolorbox}%
  \begin{beamercolorbox}[colsep=1.5pt]{lower separation line head}
  \end{beamercolorbox}
}
\setbeamertemplate{headline}[myheadline]

\makeatletter
    \newenvironment{withoutheadline}{
        \setbeamertemplate{headline}[default]
        \def\beamer@entrycode{\vspace*{-\headheight}}
    }{}
\makeatother

\AtBeginSection[]
{
  \begin{frame}<beamer>{大纲}
    \tableofcontents[currentsection]
  \end{frame}
}

\usepackage{xcolor}
\usepackage{listings}
\lstset{
	basicstyle = \ttfamily,
  showstringspaces = false,
	frame = lrtb,
	% numbers = left,
	captionpos = t,
	breaklines = true
  basewidth = 0.5em,
  % linewidth = 8cm,
}

\usepackage{amsmath}
\DeclareMathOperator*{\argmax}{arg\,max}
\DeclareMathOperator*{\argmin}{arg\,min}

\usepackage{fontspec}
\usepackage{newunicodechar}
\setmonofont{Consolas}
\newfontfamily{\fallbackfont}{Menlo}[Scale=MatchLowercase]
\DeclareTextFontCommand{\textfallback}{\fallbackfont}
\newunicodechar{∀}{\textfallback{∀}}

\usepackage{mathastext}
\MTfamily{\ttdefault}\Mathastext % reload, but use \ttdefault

\usepackage{booktabs}
\newcommand{\ty}{{:}}
\newcommand{\su}{{<:}}
\newcommand{\at}{\texttt{@}}
\newcommand{\tlb}{\texttt{[}}
\newcommand{\trb}{\texttt{]}}

% \newunicodechar{ϱ}{\textfallback{ϱ}}
\begin{document}
% \begin{withoutheadline}
\begin{frame}
\titlepage
\end{frame}
% \end{withoutheadline}
% \nottitlepagetrue

\begin{frame}
\frametitle{Contents}
\tableofcontents
\end{frame}
\section{Introduction}
\begin{frame}
    \frametitle{Motivation}
    
    Distributed computing is often achieved by networked computers which communicate and coordinate their actions by passing messages to each other. Each processor has access to its own private memory (either on different machines or on isolated segments of the same machine).

    There exists infrastructures like Message Passing Interface (MPI), but explicit communication is not easy to write and read.
\end{frame}

\begin{frame}[fragile]
\frametitle{Goals}

We would like to build a language without explicit communication and that language eventually compiles to MPI.

\begin{itemize}
  \item Encode the location information in the type system, i.e. for type \verb|T|, there exists type \verb|T @ n| for data located at the memory owned by process \verb|n|;
  \item Provide built-in polymorphic functions with \emph{ad hoc} polymorphism, which handles all communication work;
  \item Let user define their own polymorphic functions with parametric polymorphism.
\end{itemize}

Nevertheless, let's talk about the formal definition of the problem to solve after the formal definition of our language.

\end{frame}

\begin{frame}[fragile]
\frametitle{Principles}

We make heavy use of bounded quantification (polymorphism constrained to subtypes of a type) in system $\mathsf{F_{<:}}$.

Bounded quantification make some assumptions about the universal type \verb|X|, formally \verb|λX<:T.t|, so that we can use these assumptions to operate data in the body \verb|t|.

\begin{example}
    \verb|  f = λX<:{a:Nat}. λx:X. {asucc=succ(x.a)}|
    \\
    \verb|▸ f : ∀X<:{a:Nat}. X -> {asucc:Nat}|
\end{example}

We base our language on pure $\mathsf{F_{<:}}$ (on TAPL page 392) with basic types (string, record, boolean, integers and floating points) and features (list, ascribing, existential types, let-binding and fix).

The full definition is available in our later report.

\end{frame}

\section{Formalities}

\begin{frame}[fragile]
  \frametitle{Syntax (1)}
  \centering
  \begin{tabular}{llr}
      t ::= &  & terms: \\
      & $x$ & \textit{variable} \\
      & $l$ & \textit{literal} \\
      & $t~\at~p$ & \textit{distribution} \\
      & $if~t~then~t~else~t$ & \textit{condition} \\
      & $t~o~t$ & \textit{binary expression} \\
      & $\texttt{λ}x\ty T.t$ & \textit{abstraction} \\
      & $t~t$ & \textit{applicaton} \\
      & $\texttt{λ}X\su T.t$ & \textit{type abstraction} \\
      & $t~\tlb T\trb$ & \textit{type applicaton} \\
      & $fix~t$ & \textit{fix} \\
      & $t~as~T$ & \textit{ascription} \\
      & $cons~t~t$ & \textit{list constructor} \\
      & $isnil~t$ & \textit{test for empty list} \\
      & $hd~t$ & \textit{head of a list} \\
      & $tl~t$ & \textit{tail of a list} \\
  \end{tabular}

\end{frame}

\begin{frame}[fragile]
  \frametitle{Syntax (2)}
  \centering
  \begin{tabular}{llr}
      l ::= &  & literals: \\
      & $\mathtt{\cdots, -1, 0, 1, \cdots}$ & \textit{il} \\
      & $\mathtt{1.0, 2.5, 3.14159, \cdots}$ & \textit{fl} \\
      & $\texttt{"hello"}$ & \textit{sl} \\
      & $true, false$ & \textit{bl} \\
      & $unit$ & \textit{ul} \\
      & $\texttt{[]}~as~List~T$ & \textit{ll} \\
      & $\mathtt{\tlb t_1, t_2, \cdots\trb}$ & \textit{ll} \\
      o ::= & \verb|+, -, *, /, <, =, >| & operators \\
      v ::= &  & values: \\
      & $\texttt{λ}x\ty T.t$ & \textit{abstraction value} \\
      & $\texttt{λ}X\su T.t$ & \textit{type abstraction value} \\
      & $\texttt{(il|fl|sl|bl|ul) @ p}$ & \textit{distribution value} \\
      & $\mathtt{\tlb v_1, v_2, \cdots\trb}$ & \textit{list value} \\
      p ::= & $\mathtt{0, 1, 2, \cdots}$ & positions \\
  \end{tabular}

\end{frame}

\begin{frame}[fragile]
  \frametitle{Syntax (3)}
  \centering
  \begin{tabular}{llr}
      T ::= &  & types: \\
      & $X$ & \textit{type variable} \\
      & $Top$ & \textit{maximum type} \\
      & $T\to T$ & \textit{maximum type} \\
      & $\texttt{∀}X\su T.T$ & \textit{universal type} \\
      & $T\texttt{ @ }p$ & \textit{distribution type} \\
      & $List~T$ & \textit{list type} \\ 
      & $Int, Float, Bool, String, Unit$ & \textit{base types} \\
  \end{tabular}

\end{frame}

\begin{frame}
  \frametitle{Subtyping \fbox{$\texttt{Γ} \vdash S <: T$} and Typing \fbox{$\texttt{Γ} \vdash t : T$}}

  We omit subtyping and typing rules from purefsub and common extensions like ascription, conditional, let, fix and lists.
  
  \begin{equation}
      \mathtt{\texttt{Γ} \vdash T\texttt{ @ }l <: T}
      \tag{\textsc{S-Dist}}
      \label{sdist}
  \end{equation}

  \begin{equation}
    \frac{\mathtt{\texttt{Γ} \vdash t : T}}{\mathtt{\texttt{Γ} \vdash t \texttt{ @ } l : T \texttt{ @ } l}}
    \tag{\textsc{T-Dist}}
    \label{tdist}
  \end{equation}

  \begin{equation}
    \frac{\mathtt{\texttt{Γ} \vdash t_1 : T,\texttt{Γ} \vdash t_2 : T, T = (Int|Float)}}{\mathtt{\texttt{Γ} \vdash t_1~(+|-|*|/)~t_2 : T}}
    \tag{\textsc{T-Binary1}}
    \label{tbinary1}
  \end{equation}

  \begin{equation}
    \frac{\mathtt{\texttt{Γ} \vdash t_1 : T,\texttt{Γ} \vdash t_2 : T, T = (Int|Float)}}{\mathtt{\texttt{Γ} \vdash t_1~(>|=|<)~t_2 : Bool}}
    \tag{\textsc{T-Binary2}}
    \label{tbinary2}
  \end{equation}

  \begin{equation}
      \frac{\mathtt{\texttt{Γ} \vdash t_i : T}}{\mathtt{\texttt{Γ} \vdash [t_1, t_2, \cdots] : List~T}}
      \tag{\textsc{T-List}}
      \label{tlist}
  \end{equation}

\end{frame}

\begin{frame}
  \frametitle{Evaluation \fbox{$t \longrightarrow t'$}}
  
  We omit evaluation rules from purefsub and common extensions like ascription, conditional, let, fix and lists.
  
  \begin{equation}
    \frac{t \longrightarrow t'}{t\texttt{ @ }p\longrightarrow t'\texttt{ @ }p}
  \tag{\textsc{E-At}}
  \label{eat}
  \end{equation}

  \begin{equation}
    l \longrightarrow l\texttt{ @ }p
  \tag{\textsc{E-Scatter}}
  \label{escatter}
  \end{equation}

  \begin{equation}
    l\texttt{ @ }p\texttt{ @ }q \longrightarrow l\texttt{ @ }q
  \tag{\textsc{E-Move}}
  \label{emove}
  \end{equation}

  \begin{equation}
    t_1~o~t_2 \longrightarrow \textrm{equivalence in OCaml}
  \tag{\textsc{E-Binary}}
  \label{ebinary}
  \end{equation}

  \begin{equation}
    \frac{\texttt{Γ} \vdash v : T \texttt{@} l}{(\texttt{λ} X <: T. \texttt{λ} x : X . t) v \longrightarrow [X \mapsto T \texttt{@} l, x \mapsto v] t}
    \tag{\textsc{E-Sugar}}
    \label{esugar}
  \end{equation}
  
  \end{frame}

\begin{frame}[fragile]
  \frametitle{Implementation Details of Evaluation}

  The internal binding to MPI is not relevant to our central topic, but it may be interesting to talk about.

\begin{example}
\verb|mpiexec -n 8 ./f test.f|
\end{example}

  \begin{enumerate}
    \item In the typing stage, one of the processor computes the type;
    \item At the beginning of the evaluation stage, each processor obtains the same AST;
    \item At each evaluation step, each processor
    \begin{enumerate}
      \item gives a dummy value if this evaluation is irrelevant to it;
      \item computes the actual value if this evaluation is relevant, and depending on the situation, fetch data from other processes.
    \end{enumerate}
  \end{enumerate}

\end{frame}

\begin{frame}
  \frametitle{Related Attempts}

  It is also possible to define a parallelization scheme with a light-weight multithreading spawn/join style.
  \begin{block}{Syntax, Typing and Evaluation}
    \centering
  \begin{tabular}{ll}
    t ::= & ..., \texttt{spawn t}, \texttt{join t}, \texttt{domain [T]} \\
    v ::= & ..., $\mathtt{domain_t}$ \\
    T ::= & ..., \texttt{Domain T} \\
  \end{tabular}
    $$
      \frac{\texttt{Γ}\vdash t:Unit\to T}{\texttt{Γ}\vdash spawn~t:Domain~T}\quad
      \frac{\texttt{Γ}\vdash t:Domain~T}{\texttt{Γ}\vdash join~t:T}
    $$
    $$
      spawn~\texttt{λ}x.t\longrightarrow domain_t\quad join~domain_t\longrightarrow t
    $$
    $$
      \frac{t\longrightarrow t'}{spawn~t\longrightarrow spawn~t'}\quad
      \frac{t\longrightarrow t'}{join~t\longrightarrow join~t'}
    $$
  \end{block}

\end{frame}

\section{Properties}
\begin{frame}
    \frametitle{Preservation}

    \begin{theorem}[Preservation]
      \upshape
      If $\texttt{Γ}\vdash t : T$ and $t\longrightarrow t'$, then $\texttt{Γ}\vdash t' : T$.
    \end{theorem}
    \begin{proof}
      By induction on a derivation of $\texttt{Γ}\vdash t : T$, paying attention to:
      \begin{itemize}
        \item \ref{sdist}
        \item \ref{tdist}
        \item \ref{tbinary1}
        \item \ref{tbinary2}
      \end{itemize}
    \end{proof}

    The full proof is available in our report.
\end{frame}
\begin{frame}
    \frametitle{Progress}

    \begin{theorem}[Progress]
      \upshape
      If $t$ is a closed, well-typed term, then either $t$ is a
      value or else there is some $t'$ with $t\longrightarrow t'$.
    \end{theorem}
    \begin{proof}
      By induction on a derivation of $\texttt{Γ}\vdash t : T$, paying attention to:
      \begin{itemize}
        \item \ref{sdist}
        \item \ref{tdist}
        \item \ref{tbinary1}
        \item \ref{tbinary2}
      \end{itemize}
    \end{proof}

    The full proof is available in our report.

\end{frame}
\begin{frame}
  \frametitle{Problem Solving}

  \begin{theorem}[Distribution Invariance]
    \upshape
    Suppose a function $f: \texttt{∀}X_i. (X_1\su T_1)\to (X_2\su T_2)\to\cdots (X_n\su T_n)$ is defined in this language, where $T_i$ are types of ``distributable'' data, then $f$ gives the same output regardless of the input types (i.e. positions).
  \end{theorem}
  \begin{proof}
    By induction on both the number of arguments and the evaluation process, paying attention to:
    \begin{itemize}
      \item \ref{ebinary}
      \item \ref{eat}
      \item \ref{emove}
      \item \ref{escatter}
      \item \ref{esugar}
    \end{itemize}
  \end{proof}

  The full proof is available in our report.

\end{frame}
\section{Examples}

\begin{frame}[fragile]
\frametitle{I. Distribution Types}

\begin{example}[Distribution Types]
    \verb|  a = 1 @ 1|
    \\
    \verb|▸ a : Int @ 1|
    \\
    \verb|  b = 2 @ 2|
    \\
    \verb|▸ b : Int @ 2|
    \\
    \verb|  b' = 2|
    \\
    \verb|▸ b' : Int @ 7|
    \\
    \verb|  list = [4 @ 4, 5 @ 5, 6 @ 6]|
    \\
    \verb|▸ list : List Int|
    \\
    \verb|  autolist = [7, 8, 9, 10]|
    \\
    \verb|▸ autolist : List Int|
\end{example}

\end{frame}

\begin{frame}[fragile]
    \frametitle{II. Built-in Polymorphic Functions}

    \begin{example}[Built-in Polymorphic Operators]
        \verb|  c = a + b @ 3|
        \\
        \verb|▸ c : Int @ 3|
        \\
        \verb|  t = b > a|
        \\
        \verb|▸ t : Bool @ 2|
    \end{example}

\end{frame}
\begin{frame}[fragile]
    \frametitle{III. User-defined Polymorphic Functions}
\begin{example}[User-defined Polymorphic Functions]
    \verb|  square = λX<:Int. λx:X. x * x|
    \\
    \verb|▸ square : ∀X<:Int. X -> Int|
    \\
    \verb|  min = λX<:Int. λY<:Int. λx:X. λy:Y.|
    \\
    \verb|        if (x > y) then y else x|
    \\
    \verb|▸ min : ∀X<:Int. ∀Y<:Int. X -> Y -> Int|
\end{example}

\end{frame}

\begin{frame}[fragile]
    \frametitle{IV. User-defined Functional}

    \begin{example}[User-defined Functional]
        \verb|▸ map : ∀X. ∀Y. (X -> Y) -> List X -> List Y|
        \\
        \verb|▸ reduce : ∀X. (X -> X -> X) -> List X -> X -> X|
        \\
        \verb|  squared_list = map [Int] [Int] square list|
        \\
        \verb|▸ squared_list : List Int|
        \\
        \verb|  value = reduce [Int] min autolist 0|
        \\
        \verb|▸ value : Int|
    \end{example}    

\end{frame}

\section{Conclusion}

\begin{frame}
    \frametitle{Summary}
  
    \begin{itemize}
      \item Reviewed system $F_{<:}$ and its properties
      \item Designed and implemented a language based on system $F_{<:}$ that can abstract distribution with type systems and eventually compiles to MPI
      \item Proved the soundness of such a language
      \item Provided concrete examples
    \end{itemize}

\end{frame}

\begin{frame}
    \frametitle{Deficiencies and Possible Improvements}
  
    \begin{itemize}
      \item Since all ad-hoc polymorphisms are hard-coded in the language, users cannot define their own dispatch behaviors
    \end{itemize}

\end{frame}

\begin{frame}
\frametitle{Acknowledgement}

We together carried out the formal definitions; Wang implemented the lexer and parser; Zhu implemented the type checker and evaluator; Tan implemented the binding to MPI and made this beamer slide.

We thank Prof. Hu, Zhao and Xiong for guidance.

\end{frame}

\end{document}